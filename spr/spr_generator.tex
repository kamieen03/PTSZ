\documentclass[12pt]{article}
\usepackage{hyperref}
\usepackage{amsmath}
\usepackage[margin=1.0in]{geometry}
\usepackage[utf8]{inputenc}
\usepackage[T1]{fontenc}
\usepackage[polish]{babel}
\usepackage{polski}

\title{Generator instancji}
\author{Kamil Burdziński 132200}
\date{}

\begin{document}
\maketitle

\section{P - czas trwania}
$p_i$ są generowane z rokładu normalnego o średniej 27 i odchyleniu standardowym 13. Zadania długości 1 i 2
mogłyby się okazać zbyt proste do uszeregowania i niemal wcale nie wpływać na wartość tardiness, dlatego
minimalne $p$ zostało ustalone na 3. Aby uniknąć mało prawdopodobnej sytuacji, w której niewielki podzbiór bardzo
długich zadań decyduje o wartości tardiness, maksymalna wartość $p$ została ustalona na 60.

\section{R - ready time}
Po wygenerowaniu $p_i$, ich wartości są sumowane, dzielone przez $4$ i zapisywane w zmiennej $S$.
Następnie generowana jest pięcioelementowa tablica wartości z rozkładu jednostajnego
na przedziale $[5, \tfrac{4}{5}*S]$. Z tak utworzonej tablicy, dla zadania $i$,
losowana jest liczba $m$. Wówczas wartość $r_i$ losowana jest z rozkładu normalnego 
o średniej $m$ i odchyleniu standardowym $\tfrac{S}{100}$. Wartość ujemnych $r_i$ jest 
ustawiana na $0$.
\section{D - delay time}
Wartość $d_i$ jest losowana z rozkładu jednostajnego na przedziale 
$[p_i+r_i+10; p_i+r_i+40)$. \\

Po wygenerowaniu parametrów, są one normalizowane tak, aby $min\{r_i\}=0$, przy niezmienionej
długości przedziałów $[r_i; r_i+d_i]$
Dokładniej obliczna jest wartość $R = min\{r_i\}$ i wykonywane są podstawienia:
$r_i = r_i - R$, $d_i = d_i - R$.



\end{document}

